% Muster für die Seminarausarbeitung
% HPI Potsdam
\documentclass[11pt, a4paper]{article}

\usepackage[ngerman]{babel}
\usepackage[utf8]{inputenc} %Korrekte Kodierung der Umlaute nach UTF-8
\usepackage[T1]{fontenc} %Korrekte Kodierung der Umlaute nach UTF-8
\usepackage{amsfonts}
\usepackage{amssymb}
\usepackage{csquotes}
\usepackage{epsfig}   % Zum Einbinden von Bildern
\usepackage{url}      % Korrekter Satz von URLs
\usepackage{soulutf8}
\usepackage{color}    % Verwendung von Farben
\usepackage{listings} % Korrekter Satz von Listings und Quellcode

%Hilfs-Fonts - ohne Serifen (hier für Tabellen)
\newfont{\bib}{cmss8 scaled 1040}
\newfont{\bibf}{cmssbx8 scaled 1040}

\definecolor{lightgray}{gray}{0.85}

%Seitenformat-Definitionen
\topmargin0mm
\textwidth147mm
\textheight214mm
\evensidemargin5mm
\oddsidemargin5mm
\footskip19mm
\parindent=0in

\begin{document}          

\begin{titlepage}
  \begin{center} 
    \mbox{}
    \vspace{1cm}
    
    {\huge Ausarbeitung \\[1em] {\LARGE Gruppe 4}}  
        
    \vspace{5cm}
    
    Seminararbeit im Seminar \\[1em]
    {\large \sc Unternehmensanwendungen: Prozesse, Modelle und Implementierung} \\[1em]
    Sommersemester 2017 \\[1em]
    Hasso-Plattner-Institut für Softwaresystemtechnik GmbH \\[1em]
    Universität Potsdam
    
    \vspace{3cm}
    
		vorgelegt von
		
    \vspace{1em}
    
		{\Large Jonas Umland} \\
		{\Large Tom Schwarzburg}\\
		{\Large Laura Yrjänä}\\
		{\Large Justus Eilers}
		
    \vspace{3em}
    
    26.~Juli 2017
  \end{center}
\end{titlepage}


\setcounter{page}{1}

% Zweite Seite = Kurzzusammenfassung


% Dritte Seite = Inhaltsverzeichnis
\tableofcontents 

\newpage

% Vierte Seite = Hier geht's eigentlich richtig los
\section{Grundbegriffe des Rechnungswesens}

\subsection{Balance Sheet (Bilanz)}
In der Bilanz wird die Vermögenslage dargestellt. Hierbei gibt es zwei Dinge zu unterscheiden.
Erstens: Woher kommt das Geld? Diese Mittel bezeichnet man als Passiva. Es besteht die Möglichkeit, dass das Unternehmensvermögen aus Eigen- oder auch aus Fremdkapital kommt. Auch fallen Rückstellungen und Verbindlichkeiten in die Kategorie der Passiva. Rückstellungen können zum Beispiel Steuerrückstellungen oder Pensionsrückstellungen sein. Verbindlichkeiten können gegenüber Anlegern, Geldinstituten oder Privatpersonen auftreten.
Zweitens: Aktiva beschreiben, was mit dem Geld gemacht wird. Geld kann entweder angelegt werden oder sich im Umlauf befinden. So gibt es bei den Aktiva einmal das Anlagevermögen und zum Anderen das Umlaufvermögen. Vermögen lässt sich in Immaterielle Dinge oder Sachen anlegen. Es ist auch möglich nur das Geld “arbeiten” zu lassen. Diese Form der Anlage bezeichnet man dann als Finanzanlage. Umlaufvermögen wird aufgeschlüsselt in Vorräte, Forderungen, Wertpapiere und Liquide Mittel.
Durch diese Aufschlüsselung ist es nun möglich zu sehen ob das Unternehmen Gewinn oder Verlust macht. Außerdem ist die Bilanz nötig um erkennen zu können woher das Geld kommt und wohin es investiert wird.

\subsection{Profit \& Loss Statement, P\& L (Gewinn- und Verlustrechnung)}
In der Gewinn- und Verlustrechnung geht es darum Einnahmen und Ausgaben aufzurechnen, damit man erfahren kann mit welchem Aufwand welcher Ertrag erreicht werden konnte. Dazu gibt es zwei verschiedene Verfahren. Das Erste ist das Umsatzkostenverfahren. Bei diesem Verfahren wird von den Erlösen die Kosten des Umsatzes und die Kosten der Funktionsbereiche abgezogen.
Bei dem Gesamtkostenverfahren verrechnet man den Erlös mit Bestandsänderungen und zieht dann die gesamten Kosten ab

\subsection{Cash Flow Statement (Kapitalflußtechnung)}
Jetzt bei dem Cash Flow Statement werden zusätzlich zu den Einnahmen und Ausgaben, die bei der P\& L-rechnung berücksichtigt werden zusätzlich noch Aufwände, die keine Auszahlungen sind, Auszahlungen, die keine Aufwände sind, Erträge, die keine Einzahlungen sind und Einzahlungen, die keine Erträge sind berücksichtigt. In diese Kategorien fallen zum Beispiel Abschreibungen, Investitionen und Desinvestitionen.
Das Cash Flow Statement erlaubt es Aussagen über die Fähigkeit des Unternehmens zum bezahlen seiner Rechnungen zu treffen.
 
\newpage
%
\section{Aufbau und Inhalt der Seminararbeit}
\label{sec_aufbau}

Im vorangegangenen Kapitel hatten wir bereits die Gliederung einer Seminararbeit kurz vorgestellt und erläutert, welche inhaltlichen Punkte in der \glqq Einleitung\grqq\, behandelt werden sollten.
Die folgenden Abschnitte skizzieren inhaltlich die übrigen der bereits genannten Gliederungspunkte.

\smallskip

Wie Sie sehen, ist es wichtig, Überleitungen zwischen den einzelnen Kapiteln anzulegen.
Überleitungen erleichtern es dem Leser, den Überblick über das Thema zu behalten und das jetzt folgende Kapitel richtig in den Gesamtzusammenhang einzuordnen.
Enthält das Kapitel mehrere Unterkapitel, empfiehlt es sich diese hier an dieser Stelle kurz zusammenzufassen.

\smallskip

Für die vorliegende Musterausarbeitung könnte diese Zusammenfassen etwa folgendermaßen aussehen:
Zunächst wird auf die Bedeutung verwandter Arbeiten und des wissenschaftlichen Hintergrunds eingegangen, bevor anschließend der eigene Lösungsansatz für die gestellte Aufgabe vorgestellt wird. 
Danach folgen einige Tipps, was bei der Darstellung der Evaluation als Beleg für die Gültigkeit und die Qualität der gewählten Problemlösung zu beachten ist, gefolgt von Hilfestellungen für die Diskussion der erzielten Ergebnisse.
Im letzten Punkt wird auf die Bedeutung der Zusammenfassung und des Ausblicks auf anschließende Forschungsarbeiten eingegangen.


\subsection{Verwandte Arbeiten und wissenschaftlicher Hintergrund (Related Work)}
%%
Hier sind vor allem zwei inhaltliche Punkte zu berücksichtigen:
\begin{itemize}
\item {\bf Vorarbeiten und Grundlagen, die zum Verständnis der Arbeit notwendig sind}\\
Keine bzw. kaum eine Arbeit beginnt als \glqq tabula rasa\grqq , d.h. meist bauen wir auf  vorhandenen Grundlagen bzw. Vorarbeiten auf.
Die zum Verständnis der eigenen Arbeit notwendigen wissenschaftlichen Grundlagen und Voraussetzungen müssen in diesem Kapitel skizziert bzw. zusammengefasst werden.
Dabei sollte man vom durchschnittlichen Kenntnisstand eines Informatikers ausgehen, d.h. Allgemeinplätze und allzu Grundlegendes hat hier nichts zu suchen.
Genauso soll hier nicht notwendigerweise eine kompletter Wissenschaftszweig in epischer Tiefe ausgebreitet werden, sondern lediglich die zum Verständnis notwendigen Teilbereiche in skizzenhafter Form und mit Angabe von weiterführenden Literaturhinweisen zusammengefasst werden. 

\smallskip

Zum Beispiel könnten an dieser Stelle die Grundlagen und Vorzüge von Linked Open Data und deren Bezug zur vorliegenden Arbeit erläutert werden.

\smallskip

\item {\bf Alternative Ansätze und Forschungsarbeiten zum Thema}\\
Besonders wichtig ist es, spezielle Vorarbeiten und alternative Lösungsansätze zum behandelten Thema darzulegen.
Gibt es zu der von Ihnen gewählten Problemstellung bereits alternative Lösungen, die einen anderen oder vergleichbaren Ansatz verfolgen? Wie unterscheiden sich diese Lösungen von Ihrem eigenen Ansatz, wo liegen Vor- und Nachteile des jeweiligen Ansatzes? Grenzen Sie ihren eigenen Forschungsansatz von den alternativen referenzierten Ansätzen argumentativ ab.

\smallskip

So könnten Sie erwähnen, dass Ihr eigener Ansatz z.B. von den selben Voraussetzungen ausgeht wie Ansatz XY, im Gegensatz zu diesem aber auf dem Einsatz von Linked Data Technologien beruht, und daher bessere / genauere / umfangreichere Ergebnisse erzielt. (Bitte wiederholen Sie diese Argumentation nicht. Sie dient nur als illustrierendes Beispiel.)

Wichtig ist, dass Sie jede der vorgestellten, alternativen wissenschaftlichen Arbeiten 
\begin{itemize}
\item korrekt zitieren (Bibliografie),
\item kurz die wichtigsten Ergebnisse bzw. Strategien skizzieren und
\item diese (kurz und knapp) in Zusammenhang mit ihrer eigenen Arbeit stellen. 
\end{itemize}
Wie unterscheidet sich der eigene Ansatz von den vorgestellten Arbeiten? 
Warum ist der eigene Ansatz eventuell erfolgsversprechender? 

\end{itemize}

\subsection{Eigener Ansatz zur Lösung der gestellten Aufgabe (Method and Approach)}
%%
Hier haben Sie die Freiheit, Ihren eigenen Forschungsarbeiten angemessen viel Raum zur Verfügung zu stellen.
Achten Sie dabei auf einen logischen Aufbau der Darstellung, d.h. Grundlegendes stets zuerst.
\begin{itemize}
\item Wie sind Sie vorgegangen?
\item Wo gab es (welche) Probleme?
\item Wie wurden diese gelöst?
\item Schreiben Sie in verständlicher Weise und drücken Sie sich dabei jeweils möglichst präzise, d.h. unmissverständlich aus (vgl. Kap.~\ref{sec_stil})
\item Verwenden Sie Abbildungen, Tabellen und Beispiele, um Ihren Ansatz besser zu erläutern. Stellen Sie ein Softwaresystem vor, dann hilft eine Architekturskizze. Stellen Sie einen Workflow vor, dann hilft ein Workflow-Diagramm.
\item Setzen Sie kein Wissen als implizit vorhanden voraus, sondern sprechen Sie explizit alle Probleme und wichtigen Fakten an.
\item Wichtig: Was Sie hier nicht beschreiben, können wir nicht bewerten!
\end{itemize}
Bedenken Sie dabei stets, dass ein Leser nicht dasselbe Wissen besitzen kann wie Sie und dass Sie ihm deshalb alle ihre Ergebnisse erklären müssen.

\smallskip

Verwenden Sie bei der Darstellung Ihres Lösungsansatzes eine möglichst einfache Sprache.
Vermeiden Sie zahlreiche Schachtelungen und Nebensätze. Der dargestellte Sachverhalt ist meist bereits hinreichend komplex. 
Die zu seiner Darstellung verwendete Sprache sollte sollte den Zugang für den Leser nicht auch noch zusätzlich erschweren.


\subsection{Evaluation des wissenschaftlichen Beitrags (Evaluation)}
%
Viele wissenschaftliche Aufgabenstellungen erfordern den Nachweis der Qualität bzw. der Effizienz des vorgestellten Lösungsansatzes, d.h. das entwickelte Verfahren bzw. die vorgestellte technische Lösung muss objektiv getestet und anschließend bewertet werden.
Man unterscheidet hier grundsätzlich zwischen quantitativer und qualitativer Evaluation.
Während in der qualitativen Evaluation oft menschliche Testpersonen die Qualität einer Lösung entsprechend vorgegebener Qualitätskriterien beurteilen müssen, stützt sich die quantitative Evaluation auf vorgegebene Testdatensätze und einen Soll-/Ist-Vergleich.
Testdatensätze (auch Benchmarks) enthalten korrekte, oft manuell verifizierte Ergebnisse zu vorgegebenen Aufgabenstellungen (Ground Truth).
Diese werden quantitativ mit den Ergebnissen des von Ihnen implementierten Verfahrens verglichen und in Form von statistischen Kenngrößen (z.B. Recall, Precision und F$_1$-Measure) beschrieben.

\begin{itemize}
\item Verwenden Sie (wenn möglich) vorgegebene Standard Benchmarks, um eine möglichst breite Vergleichbarkeit mit alternativen Lösungsansätzen bzw. Systemen zu ermöglichen und die in der Wissenschaft notwendige Nachvollziehbarkeit zu gewährleisten.
\item Wenn Sie eigene Testdaten bzw. Benchmarks zusammenstellen, stellen Sie diese öffentlich (im WWW) zur Verfügung, um zukünftig bessere Vergleichberkeit und Nachvollziehbarbeit zu gewährleisten.
\end{itemize}

%\bigskip

Nicht jede Aufgabenstellung ist für eine solche Evaluation geeignet. Theoretische und mathematische Aufgabenstellungen resultieren oft in Lösungen, deren Gültigkeit mit mathematischen Methoden (mathematischen Beweisverfahren) belegt werden muss.



\subsection{Diskussion der erzielten Ergebnisse (Discussion)}
%%
In diesem Kapitel sollten Sie Ihre erzielten Ergebnisse (möglichst vollständig) präsentieren und deren Qualität diskutieren.
Sollte der zur Verfügung stehende Platz für eine vollständige Darstellung im Rahmen der Arbeit nicht ausreichen, stellen Sie an dieser Stelle die wichtigsten Resultate dar und verweisen Sie auf eine vollständige Version, die Sie z.B. auf einer entsprechenden Webpage bereithalten. 
Dabei sollten (falls jeweils zutreffend) folgende Fragen beantwortet werden:
\begin{itemize}
\item Was wurde erreicht, was kann noch verbessert werden bzw. wo gibt es noch offene (evtl. aus Zeitgründen nicht implementierte) Punkte?
\item Sind die erzielten Ergebnisse objektiv? Gibt es Gründe, daran zu zweifeln?
\item Warum ist der eigene Ansatz besser/schlechter als die zum Vergleich herangezogenen?
\item Welche Vorbedingungen könnten verändert werden, um eventuell bessere Ergebnisse zu erzielen?
\item Wenn die Evaluation nicht aussagekräftig genug ist, wie könnte man sie noch verbessern?
\item Was haben Sie aus dem Seminar mitgenommen (z.B. Wo liegen die Vorteile der im Seminar behandelten wissenschaftlichen Methoden und Verfahren?)
\end{itemize}
Der Leser sollte nach diesem Kapitel davon überzeugt sein, dass Sie Ihre Arbeit gut gemacht haben und alle offensichtlich aufgetretenen Fragen nach bestem Wissen und Gewissen beantwortet haben.

\subsection{Zusammenfassung und Ausblick (Conclusion and Outlook)}

In diesem Abschnitt werden die erzielten Ergebnisse noch einmal kurz zusammengefasst und es wird ein kurzer Ausblick auf weiterführende Forschungs- und Entwicklungsarbeiten gegeben, die sich im Zuge Ihrer Arbeiten ergeben haben (vgl. Kap.~\ref{sec_conclusion}). Hier können Sie z.B. ausführen, welche Arbeiten Sie aus Zeitgründen nicht mehr umsetzen konnten, aber für wichtig oder sinnvoll erachten. 
Unterscheiden Sie, welche Verbesserungsmaßnahmen kurzfristig bzw. langfristig vorgenommen werden könnten und versuchen Sie den dadurch erzielten Gewinn bzw. Vorteil zu quantifizieren.
\newpage
%
\section{Traditionelle Architektur einer Persistenzschicht}


\newpage
%
\section{In-memory basiertes Rechnungswesen}
Beim Rechnungswesen einer Firma vor In-Memory Systemen gab es Accounting Documents, wo zum Beispiel die Belegköpfe und -körper dabei sind, aber auch sekundäre Indizes. Weiterhin gab es bereits materialisierte Aggregate und Views, die aufwändige Queries schon zu einem Zeitpunkt vorberechnen, damit man diese dann als Cache benutzen kann. Außerdem gibt es natürlich die Change History, also die Änderungen zwischen dem OLTP- und dem OLAP-System. Dadurch, dass die Accounting Documents, also die Standardtabellen auch noch jeweils extra auf dem OLAP- und auf dem OLTP-System gespeichert sind, hat man dadurch die Daten bereits doppelt und muss diese für die materialisierten Views und Aggregate noch öfter abspeichern. Man halt also viele redundante Datensätze, die jedoch nötig waren, um überhaupt Analysen auf den Daten in einer annehmbaren Zeit durchführen zu können.
\\
Weiterhin waren die Daten in dem Data Warehouse(OLAP-System) nicht aktuell, es kann also öfters passieren, dass man die Analyse durchführt und die jüngsten Daten in der Datenbank schon über einen Monat alt sind.
Bei einem In-Memory Datensystem sollen die Daten redundanzfrei gespeichert werden. Man soll also nur noch die Belegköpfe und -körper haben , also nur die Basistabellen und sonst nichts. Die Analyse und auch die Transaktionen sollen jedoch trotzdem deutlich effizienter ablaufen.
\\
Dadurch, dass man nur noch ein System hat, spart man sich natürlich auf lange Sicht die Kosten der mehreren Server(OLTP- und OLAP- Systeme) und braucht nur noch ein In-Memory System. Damit kann man langfristig schon an Anschaffungs- und Betriebskosten der Server sparen.
\\
Da man nur noch die Basistabellen speichert, kann man die Menge an Daten, die man sonst speichern müsste um mindestens die Hälfte verringern. Man braucht somit also nun nur noch die Hälfte an Speicherkapazität wie vorher.
\\
Weiterhin hat man nun alle Daten in nur zwei Tabellen und kann einfach so auf diese zugreifen und muss nicht mit materialisierten Views oder Aggregaten arbeiten. Die Queries werden dadurch also einfacher zu schreiben und zu verstehen. Da man die Anzahl möglicher Joins, die sehr zeitaufwendig sein können, reduziert, indem man nur noch die zwei Tabellen hat, ist es auch einfacher möglich effiziente Queries in kürzerer Zeit zu schreiben.
Da Queries also schneller laufen, siehe Aufgabe 3, und man schneller diese Queries auch effizient schreiben kann, ist es auch möglich in Echtzeit Analysen durchzuführen und diese auch in Echtzeiten zu ändern, so wie sie gerade benötigt werden.
\\
Die Queries müssen also nicht 3 Monate, bevor man etwas wissen will, materialisiert werden, damit man dann irgendwann ein Ergebnis bekommt, sondern man kann in Echtzeit die aktuellen Daten, die auch wirklich aktuell sind, da keine Trennung zwischen den OLAP und OLTP Daten mehr herrscht, analysieren und somit immer sofort auf mögliche Änderungen reagieren und sein Geschäft dementsprechend anpassen.Man kann somit mögliche Fehler verhindern oder aber auch mögliche Chancen ausnutzen. 
\\
Die Benutzung einer In-Memory Technologie kann also die Komplexität der Anwendungssoftawareschichten reduzieren, Durchführung von datenintensiven Aufgaben nahe der Datenquelle ermöglichen, die Reaktionszeiten beschleunigen und dadurch die Zeit des Benutzers sparen sowie die Verwaltungsaufgaben mit Massendaten beschleunigen.
Aus der betriebswirtschaftlichen Sicht kann ein In-Memory System dessen Benutzern helfen schnellere, informiertere Entscheidungen zu treffen und Reporting und Analyse-Anwendungen anzutreiben. Dieses hingegen kann die Business-Prozesse stärken, dass zu verbesserten Ergebnissen führen kann. Zum Beispiel die Umstellung von wöchentlicher auf eine stündliche Umsatzprognose würde die Erstellung von Echtzeit- und Produktpreismodellen ermöglichen, dass weiterhin die Rentabilität steigern könnte. Dieses erfordert natürlich auch, dass Entscheidungen schnell kommuniziert und ausgeführt werden können.
\cite{In-memory}

\newpage
%
\section{Das Literaturverzeichnis und die korrekte Zitierweise}
\label{sec_literatur}

In diesem Kapitel widmen wir uns dem korrekten Zitieren.
Zunächst einmal wird dargestellt, was zitiert werden muss, bevor wir uns der korrekten Zitierweise widmen, die -- falls Sie das Satzsystem \LaTeX \ nutzen -- durch das Bibliografie-Satzsystem Bibtex bequem umgesetzt werden kann.
Oft tritt dabei die Frage auf, wie mit Quellen aus dem Internet umgegangen werden soll. 
Insbesondere wird heute gerne Wikipedia als Nachschlagewerk benutzt.
Was dabei zu beachten ist, wird in den letzten beiden Abschnitten dieses Kapitels behandelt.

\subsection{Was wird zitiert?}
Jede Behauptung tatsächlicher Art, d.h. stets wenn Sie konkrete Werte oder Aussagen wiedergeben, gilt solange als Behauptung, bis Sie diese auch belegen können.
Ein Beleg besteht entweder aus einer korrekten Herleitung, wie z.B.~einem mathematischen Beweis, oder aber in einer Angabe der Fundstelle (Literatur oder WWW), aus der die besagte Behauptung gewonnen wurde (= bibliografische Referenz). 
Erwähnen Sie in Ihrer Arbeit Internetdienste, Programme, Sprachen, Internetstandards oder bestimmte Werkzeuge, dann belegen Sie diese beim ersten Vorkommen in Ihrem Text mit einer bibliografischen Referenz.
Im Allgemeinen wird immer stets an genau der Stelle zitiert, die es zu belegen gilt \cite{Marchionini}.

\smallskip

Achten Sie darauf, dass der Zitierhinweis stets Bestandteil des Satzes ist, d.h. der Punkt kommt erst dahinter.
Die Quellenangabe sollte auf das am Ende der Ausarbeitung vorhandene Literaturverzeichnis verweisen. 
Gegebenenfalls kann man für diesen Zweck {\em zusätzlich} Fußnoten verwenden.


\subsection{Bibtex -- das Zitiersytem von \LaTeX}

BibTeX ist ein Programm zur Erstellung von Literaturangaben und -verzeichnissen in TeX- oder \LaTeX-Dokumenten.

\smallskip

Um ein Literaturverzeichnis zu erstellen, werden aus einem \LaTeX-Dokument alle Zitatverweise herausgesucht und über eine Literatur-Datenbank dem entsprechenden Werk zugeordnet. Bei der Literaturdatenbank handelt es sich um eine Textdatei (*.bib-Datei), in der alle bekannten Angaben über ein Werk (Buch, wissenschaftliche Publikation, Webseite, etc.) in einer vorgegebenen Syntax notiert werden.

\smallskip

Die zitierten Werke werden (automatisch) sortiert und durch eine entsprechende Anweisung im \LaTeX-Dokument aufgelistet. 
Die Formatierung dieser Literaturliste ist variabel. 
Der im Dokument eingestellte BibTeX-Stil (engl. {\em style}) bestimmt, welche Angaben in welcher Formatierung dargestellt werden.

\smallskip

BibTeX ist in der Lage, auch mit sehr großen Literaturbeständen sowie mit sehr großen Dokumenten problemlos zusammenzuarbeiten. 
BibTeX hat sich daher im wissenschaftlichen Umfeld schon seit Jahren als offenes Standard-Datenformat für Literaturangaben etabliert.
Zu zahlreichen wissenschaftlichen Arbeiten können Sie daher bereits die komplette zugehörige BibTex-Zitation im WWW finden.

\smallskip

Das folgende Beispiel (entnommen aus einer BibTeX-Datei)

\begin{verbatim}
 @article{lin1973,
    author  = {Shen Lin and Brian W. Kernighan},
    title   = {An Effective Algorithm for the Travelling-Salesman Problem},
    journal = {Operations Research},
    volume  = {21},
    year    = {1973},
    pages   = {498--516},
 }
\end{verbatim}

wird durch den BibTeX-Stil {\em alphadin} in diese Ausgabe in der Literaturliste (engl. {\em bibliography}) überführt:

\bigskip
[LK73] Lin, Shen; Kernighan, Brian W.: An Effective Algorithm for the Travelling-Salesman Problem. In: Operations Research 21 (1973), S. 498--516
\bigskip

Der Befehl \verb+\cite{lin1973}+ innerhalb eines LaTeX-Dokuments wird durch die in der BibTeX-Datei mit dieser ID angegebene Referenz, im Beispiel '[LK73]', ersetzt.

\smallskip

Neben dem BibTeX-Stil {\em alphadin} gibt es den Stil {\em plain}, bei dem der Schlüssel lediglich aus Ziffern besteht, z.B. [12]. Daneben gibt es verschiedene Varianten dieser Stile, die sich hauptsächlich in der Darstellung der Literaturliste unterscheiden und oft spezifisch für verschiedene wissenschaftliche Verlage, Konferenzen und Zeitschriften sind (vgl.~\cite{bibstyle}).

Wer nicht direkt im Text zitiert hat, aber trotzdem eine Quelle im Literaturverzeichnis nennen will, die zum Erstellen der Arbeit wichtig war, tut dies durch \verb+\nocite{lin1973}+.

\nocite{lin1973}

\nocite{*} % alle Einträge werden angezeigt

\subsection{Zitieren und das Internet}
%%
Auch wichtige Quellen, die nur im Internet publiziert wurden, müssen zitiert werden.
Unterscheiden Sie bitte dabei, ob es sich lediglich um eine Web-Präsenz, wie z.B. ein Web-Portal oder eine Übersichtsseite handelt, deren Inhalt sich mit der Zeit verändern kann. 
Dies kann z.B. der Fall sein, wenn Sie die Suchmaschine Google\footnote{\url{http://www.google.com/}, zuletzt zugegriffen am 07.07.2016.} erwähnen und dazu den URL als Referenz angeben.
In diesem Fall empfiehlt es sich, die URL als Fußnote anzugeben zusammen mit dem Zeitpunkt, an dem diese zuletzt zugegriffen wurde.

\smallskip

Andererseits können Sie auch auf ein Web-Dokument verweisen, dessen Inhalt für sich selbst und der sich wahrscheinlich nicht so schnell wieder verändern wird.
Dann müssen Sie den URL des Dokuments in die Bibliographie aufnehmen.
Zur korrekten Formatierung und Silbentrennung von URLs verwenden Sie das \LaTeX-Paket {\tt URL}; dieses sorgt für ein korrektes Umbrechen am Zeilenende.
Beachten Sie hier zu jedem URL auch das Datum mit anzugeben, an dem sie den URL zuletzt erfolgreich zugegriffen haben, da Sie sich auf eine ganz bestimmte Version dieses Dokuments beziehen, dessen Inhalt sich eventuell mit der Zeit verändern könnte.


\subsection{Zitieren und die Wikipedia}
%%
Das Zitieren der Online-Enzyklopädie Wikipedia\footnote{\url{http://www.wikipedia.org/}, zuletzt zugegriffen am 07.07.2016.} wird aktuell immer noch kontrovers diskutiert.
Schuld daran ist die mangelnde Persistenz der Inhalte, d.h. im Prinzip kann jeder Benutzer den Inhalt eines Wikipedia-Artikels jederzeit willkürlich verändern, so dass dieser nicht als gesicherte Referenz herangezogen werden kann.
Auch wenn in der Wikipedia mittlerweile ein hohes Maß an Selbstkontrolle vorherrscht, sollte bei der wissenschaftlichen Bibliografie Wert auf das Prinzip der Nachvollziehbarkeit gelegt werden.
Verwenden Sie daher bitte möglichst stets gesicherte, d.h. regulär publizierte Quellenangaben.
Dies ist insbesondere dann ratsam, wenn Sie Grundlagenarbeiten und Nachschlagewerke zitieren.

\smallskip

Ein weiteres Argument gegen das Zitieren von Wikipedia liegt darin, dass es sich bei Wikipedia um eine Enzyklopädie handelt, d.h. um sogenanntes \glqq Sekundärwissen\grqq .
In einer Enzyklopädie ist üblicherweise Wissen zusammengetragen worden, das an anderer Stelle bereits schon einmal ursprünglich beschrieben wurde, d.h. in der sogenannten Primärquelle.
Wissenschaftlich valide Enzyklopädien belegen ihre Artikel jeweils über zitierte Primärquellen.
Dieses Prinzip findet auch zunehmend in der Wikipedia Verbreitung.
Daher bietet sich die Wikipedia für Sie stets als Ausgangspunkt für weitere Recherchen nach Primärquellen an, die Sie dann nach entsprechender Prüfung korrekt zitieren können.





%Hier kommt das Literaturverzeichnis
\newpage

\addcontentsline{toc}{section}{Literaturverzeichnis} % Zeile für das Inhaltsverzeichnis

\bibliography{bibfile}
\bibliographystyle{alphadin}

\end{document}
