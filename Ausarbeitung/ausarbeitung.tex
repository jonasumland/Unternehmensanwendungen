% Muster für die Seminarausarbeitung
% HPI Potsdam
\documentclass[11pt, a4paper]{article}

\usepackage{ngerman}
\usepackage[utf8]{inputenc} %Korrekte Kodierung der Umlaute nach UTF-8
\usepackage[T1]{fontenc} %Korrekte Kodierung der Umlaute nach UTF-8
\usepackage{amsfonts}
\usepackage{amssymb}
\usepackage{csquotes}
\usepackage{epsfig}   % Zum Einbinden von Bildern
\usepackage{url}      % Korrekter Satz von URLs
\usepackage{soulutf8}
\usepackage{color}    % Verwendung von Farben
\usepackage{listings} % Korrekter Satz von Listings und Quellcode


%Hilfs-Fonts - ohne Serifen (hier für Tabellen)
\newfont{\bib}{cmss8 scaled 1040}
\newfont{\bibf}{cmssbx8 scaled 1040}

\definecolor{lightgray}{gray}{0.85}

%Seitenformat-Definitionen
\topmargin0mm
\textwidth147mm
\textheight214mm
\evensidemargin5mm
\oddsidemargin5mm
\footskip19mm
\parindent=0in

\begin{document}          

\begin{titlepage}
  \begin{center} 
    \mbox{}
    \vspace{1cm}
    
    {\huge Ausarbeitung \\[1em] {\LARGE Gruppe 4}}  
        
    \vspace{5cm}
    
    Seminararbeit im Seminar \\[1em]
    {\large \sc Unternehmensanwendungen: Prozesse, Modelle und Implementierung} \\[1em]
    Sommersemester 2017 \\[1em]
    Hasso-Plattner-Institut für Softwaresystemtechnik GmbH \\[1em]
    Universität Potsdam
    
    \vspace{3cm}
    
		vorgelegt von
		
    \vspace{1em}
    
		{\Large Jonas Umland} \\
		{\Large Tom Schwarzburg}\\
		{\Large Laura Yrjänä}\\
		{\Large Justus Eilers}
		
    \vspace{3em}
    
    26.~Juli 2017
  \end{center}
\end{titlepage}


\setcounter{page}{1}

% Zweite Seite = Kurzzusammenfassung


% Dritte Seite = Inhaltsverzeichnis
\tableofcontents 

\newpage

% Vierte Seite = Hier geht's eigentlich richtig los
\section{Grundbegriffe des Rechnungswesens}

\subsection{Balance Sheet (Bilanz)}
In der Bilanz wird die Vermögenslage dargestellt. Hierbei gibt es zwei Dinge zu unterscheiden.
Erstens: Woher kommt das Geld? Diese Mittel bezeichnet man als Passiva. Es besteht die Möglichkeit, dass das Unternehmensvermögen aus Eigen- oder auch aus Fremdkapital kommt. Auch fallen Rückstellungen und Verbindlichkeiten in die Kategorie der Passiva. Rückstellungen können zum Beispiel Steuerrückstellungen oder Pensionsrückstellungen sein. Verbindlichkeiten können gegenüber Anlegern, Geldinstituten oder Privatpersonen auftreten.
Zweitens: Aktiva beschreiben, was mit dem Geld gemacht wird. Geld kann entweder angelegt werden oder sich im Umlauf befinden. So gibt es bei den Aktiva einmal das Anlagevermögen und zum Anderen das Umlaufvermögen. Vermögen lässt sich in Immaterielle Dinge oder Sachen anlegen. Es ist auch möglich nur das Geld “arbeiten” zu lassen. Diese Form der Anlage bezeichnet man dann als Finanzanlage. Umlaufvermögen wird aufgeschlüsselt in Vorräte, Forderungen, Wertpapiere und Liquide Mittel.
Durch diese Aufschlüsselung ist es nun möglich zu sehen ob das Unternehmen Gewinn oder Verlust macht. Außerdem ist die Bilanz nötig um erkennen zu können woher das Geld kommt und wohin es investiert wird.

\subsection{Profit \& Loss Statement, P\& L (Gewinn- und Verlustrechnung)}
In der Gewinn- und Verlustrechnung geht es darum Einnahmen und Ausgaben aufzurechnen, damit man erfahren kann mit welchem Aufwand welcher Ertrag erreicht werden konnte. Dazu gibt es zwei verschiedene Verfahren. Das Erste ist das Umsatzkostenverfahren. Bei diesem Verfahren wird von den Erlösen die Kosten des Umsatzes und die Kosten der Funktionsbereiche abgezogen.
Bei dem Gesamtkostenverfahren verrechnet man den Erlös mit Bestandsänderungen und zieht dann die gesamten Kosten ab

\subsection{Cash Flow Statement (Kapitalflußtechnung)}
Jetzt bei dem Cash Flow Statement werden zusätzlich zu den Einnahmen und Ausgaben, die bei der P\& L-rechnung berücksichtigt werden zusätzlich noch Aufwände, die keine Auszahlungen sind, Auszahlungen, die keine Aufwände sind, Erträge, die keine Einzahlungen sind und Einzahlungen, die keine Erträge sind berücksichtigt. In diese Kategorien fallen zum Beispiel Abschreibungen, Investitionen und Desinvestitionen.
Das Cash Flow Statement erlaubt es Aussagen über die Fähigkeit des Unternehmens zum bezahlen seiner Rechnungen zu treffen.
 
\newpage
%
\section{Aufgaben des externen und des internen Rechnungswesens}

\subsection{Aufgaben des internen Rechnungswesen}
Das interne Rechnungswesen wird auch als Controlling bezeichnet. Im Controlling wird durch Planung, Kontrolle und Koordination die Effizienz bewerteter Prozesse des Unternehmens verbessert und somit der Unternehmenserfolg maximiert. Dazu werden vor allem die Kosten- und Leistungsrechnung als auch die Investitionsrechnung herangezogen.
\\
Das interne Rechnungswesen richtet sich ausschließlich an Unternehmensinterne, unterliegt in der Regel keinen gesetzlichen Bestimmungen und ist daher optional. Je nach konkretem Rechnungszweck sind die angewandten Zeitintervalle unterschiedlich. So wird für die Investitionsrechnung ein mehrjähriger Zeitraum festgelegt, während die Kosten- und Leistungsrechnung mehrmal pro Jahr erfolgt. Hinzu kommt, dass verschiedene Unternehmensbereiche getrennt voneinander betrachtet werden können.
\\
Neben der Wirtschaftlichkeit muss das Controlling allerdings auch darum kümmern das Unternehmen mit Informationen über der Markt, Marktlücken und Trends zu analysieren und aufzubereiten. Mit all diesen Informationen soll dem Management ermöglicht werden fundierte Entscheidungen über die Zukunft des Unternehmens zu treffen.
Die Kernbereiche des internen Rechnungswesens sind Kostenrechnung, Finanzplanung und -steuerung, sowie Leistungsmessung. 
\\
Kostenrechnung schafft eine Grundlage für die wirtschaftliche Entscheidungsfindung, Wirtschaftlichkeitsberechnung und die Berichterstattung der Organisation. Finanzplanung und Beratung, wiederum quantifizieren Pläne für die Organisation, setzen Ziele für Personal und Abteilungen und folgen die Erreichung der Ziele. Mit Hilfe der Steuerung werden die Strategien von der Organisation in die Praxis umgesetzt und deren Umsetzung überwacht. Einstellen von strategischen Zielen und die Verwendung von nicht-finanziellen Indikatoren hat ebenfalls eine wichtige Rolle im internen Rechnungswesen.

\subsection{Aufgaben des externen Rechnungswesen}
Das externe Rechnungswesen bildet die finanzielle Situation des Unternehmens nach außen ab und richtet sich an Personen/Institutionen, denen ein berechtigtes Interesse an Informationen über das Unternehmen zugeschrieben wird. Dies beinhaltet Kapitalgeber, Gläubiger, das Finanzamt, Kunden und Lieferanten, den Staat und auch die interessierte Öffentlichkeit. Das Ziel des externen Rechnungswesen ist eine vergangenheitsorientierte Dokumentation, sowie die Rechenlegung. 
\\
Im Gegensatz zum internen Rechnungswesen ist das externe Rechnungswesen nicht optional und erfolgt in Form des Jahresabschlusses. In Deutschland richtet sich das externe Rechnungswesen daher nach den im HGB festgelegten gesetzlichen Bestimmungen. Für Personenunternehmen und Einzelkaufleute beinhaltete der Jahresabschluss die Bilanz und die Gewinn- und Verlustrechnung. Bei Kapitalgesellschaften kommt ein Lagebericht, der eine Darstellung der Unternehmensführung ist, sowie ein Anhang mit zusätzlichen Informationen zu Bilanz und Gewinn- und Verlustrechnung hinzu. Bestimmte Kapitalgesellschaften müssen außerdem eine Kapitalflussrechnung und einen Eigenkapitalspiegel hinzufügen.\cite{boeckler}
\\
Zu den Aufgaben des externen Rechnungswesen gehört auch die Dokumentation und Durchführung von Geschäftsvorfällen. Bei der Dokumentation müssen sowohl logistische als auch nicht-logistische Geschäftsvorfälle berücksichtigt werden. Die Durchführung von Geschäftsvorfällen beinhaltet logischer Weise nicht das Kerngeschäft sondern deren Fortführung. Das beinhaltet Zahlen und Mahnen. Nicht-logistische Geschäftsvorfälle werden vollständig durchgeführt. Nicht-logistische Vorfälle sind zum Beispiel Bargeschäfte und Finanzierung.

\subsection{Kontenmodell}
Im Kontenmodell werden Geschäftsvorfälle als Finanzbuchhaltungsbelege beschrieben. Dieser Finanzbuchhaltungsbeleg verweist auf den logistischen Beleg. Die in dem Beleg beschriebenen Positionen werden in Form von Konten beschrieben. Ein Geschäftsvorfall wird nun in eine von zwei Klassen eingeordnet. Ist es eine werterhöhende Position, ist es in der Klasse der SOLL-positionen, ist es eine werterniedrigende Position, in der Klasse er HABEN-positionen.

\subsection{Treibermodell}
Im Treibermodell werden Geschäftsvorfälle als Ursache-Wirkungsbeziehungen abgebildet. Als Wirkung auslösende Treiber werden in diesem Modell die logistischen Geschäftsvorfälle verwendet. Da es aber auch andere Gründe für Profitabilitätänderung geben kann, können auch beliebige sonstige Ereignisse der realen Welt ein Treiber sein.
Die Werte aus denen sich die Profitabilität ergibt, werden aus den Treibern errechnet. Die Formeln dafür können sich aus Standards, Erfahrungswerten aber auch rechtlichen Vereinbarungen ergeben.
\\
Die Werte aus denen sich die Profitabilität ergibt, werden aus den Treibern errechnet. Die Formeln dafür können sich aus Standards, Erfahrungswerten aber auch rechtlichen Vereinbarungen ergeben. 


\subsection{Vergleich}
Durch die Dokumentation der Vorfälle in SOLL und HABEN wird die Realität direkt abgebildet. Darüber hinaus kann der Bezug zu logistischen Objekten ohne Probleme hergestellt werden. Außerdem wird von dem konkreten logistischen Objekt abstrahiert und vereinheitlicht. Dadurch hat man die Möglichkeit konkrete Objekte, auch ohne ihren Kontext zu analysieren, betrachten zu können.
\\
Es ist keine direkte Auswirkung von einem Vorfall auf die Gesamtprofitabilität des Unternehmens zu bemerken. Um doch Informationen über Zusammenhänge von Vorfällen auf die Wirtschaftlichkeit zu erhalten, müssen nun zusätzliche Modelle erstellt werden.
\\
Ein Treibermodell ermöglicht eine wirksame Kontrolle mit Hilfe Soll-Ist-Vergleichen und eine Entscheidungsunterstützung in Form von Simulationen. Somit können sowohl die Auswirkungen als auch die Ursachen für Vorfälle ermittelt und ausgewertet werden.
\\
Wenn die Einfluss stärksten Treiber ermittelt wurden, sorgt das für eine Komplexitätsreduktion in der Planung und zur Fokussierung auf die wesentlichen Performancetreiber im Unternehmen. Außerdem kann durch die stärkere Top-Down-Orientierung und enge Verschränkung der operativen Planungszyklen mit der Strategie (strategische „Vorsteuerung“ des finanziellen Erfolgs) leichter und effizienter Entscheidungen getroffen werden. Durch das Aufschlüsseln der Informationen erhält der Analyseprozess mehr Transparenz hinsichtlich der Performancetreiber in den dezentralen Unternehmenseinheiten. Dadurch entsteht eine bessere Nachvollziehbarkeit von Planungsprämissen und Gründen für Planabweichungen.
\\
Falls dann doch mal etwas schief läuft, haben die Top-Manager einen unmittelbaren Andockpunkt für Sensitivitätsanalysen („Stress-Tests“) und Simulationsmodelle.\cite{raoul}
\\
Um wirklich den Grund für eine Einbuße in der Profitabilität zu bemerken, müssen alle Faktoren, die einen Einfluss haben erfasst worden sein. Ist dies nicht der Fall, bietet das Treibermodell unter Umständen falsche oder unvollständige Aussagen.
Außerdem ist das Erstellen eines solchen Modells sehr aufwändig. Wenn man darüber hinaus noch von der Reduktion der Einflussfaktoren gebrauch machen möchte erfordert dies extra Aufwand.



\newpage
%
\section{Traditionelle Architektur einer Persistenzschicht}
Herkömmliche Enterprise-systeme wurden in zwei Teilsysteme unterteilt. Zum einen in das OLAP(On-line Analytical Processing)- und zum Anderen das OLTP(On-line Transaction Processing)-system. In dem OLTP-system werden in Echtzeit Daten hinzugefügt, modifiziert oder gelöscht. Es entstehen also viele Transaktionen. Das OLAP-system wird verwendet um Analysen auf den Daten laufen zu lassen. Das beinhaltet zum Beispiel Anfragen, die herausfinden sollen, welches Produkt am häufigsten gekauft wurde. Entsprechend werden viele Daten gelesen und wenig geschrieben. Es entstehen also wenige Transaktionen, diese die entstehen sind jedoch komplexer.
\\
Damit die Analysen und das Geschäft auf den möglichst aktuellen Daten laufen, müssen die Daten von dem OLTP-system auf auf das OLAP-system überspielt werden. Diese Funktion übernimmt die Persistenzschicht. Der Prozess wird ETT (Extraction Transformation Transportation) genannt.
\\
Als OLTP und OLAP entwickelt wurden, analysierte man was die Aufgaben einer Datenbank größtenteils darstellen. Zum Einen gibt es Schreiboperationen, bei denen man neue Zeilen einfügt, löscht oder vorhandene Zeilen bearbeitet. Diese Operationen betreffen jeweils nur wenige Zeilen einer Datenbank, dafür werden diese sehr häufig ausgeführt. Auf der anderen Seite möchte man die Daten analysieren um “versteckte” Informationen zu finden. An einem einfachen Beispiel: Eine Firma hat täglich Millionen von Verkäufen, die sie alle in einer Datenbank mitschreibt. Aus diesen Verkaufsdaten, möchte die Firma herausfinden, in welchem Zeitraum welches Produkt am meisten verkauft wurde, damit man so auf saisonale Anforderungen reagieren und diese vorhersagen kann. Diese Analysen führt man natürlich nicht so oft aus, diese sind jedoch dafür sehr komplexe Operationen und betreffen meist einen Großteil, wenn nicht sogar alle Zeilen einer Datenbank.
Diese beiden Fälle möchte man natürlich möglichst effizient ausführen um mögliche Kosten zu vermeiden.
\\
Zu dem Zeitpunkt war es mit dem Stand der Technik jedoch nicht möglich, ein Datenbanksystem zu erstellen, welches beide Anwendungsfälle abdeckt und diese jeweils schnell und kostengünstig durchführen kann. Deshalb entschied man sich,
die beiden Anwendungsfälle in unterschiedliche Datenbanksysteme aufzuteilen: OLAP(Data Warehouse) und OLTP(operative Systeme).
\\
Das OLTP-system muss hauptsächlich Schreiboperationen durchführen die nur wenige Zeilen der Datenbank betreffen. In diesem werden die neuen Daten eingefügt und es stellt somit den aktuellen Stand der laufenden Unternehmensprozesse da. Das OLAP-system muss hingegen meistens Leseoperationen durchführen für die Analyse der Daten. Man arbeitet dann also mit ziemlich vielen Zeilen der Datenbank.
\\
Aufgrund technischer Fortschritte in der heutigen Zeit, wie zum Beispiel Multicore-Prozessoren und deutlich günstigere Preise für RAM,
ist es in der heutigen Zeit möglich, OLTP und OLAP Systeme zu einem zu vereinen.
Unterstützt wird diese Möglichkeit auch durch Untersuchungen, die herausgefunden haben, dass OLAP und OLTP Systeme nicht so genutzt werden, wie gedacht. Bei beiden System sind Schreiboperationen ein Großteil der Operationen. Beide Systeme haben also doch ähnlichere Aufgaben als vorher angenommen. In-Memory Systeme nutzen die heutige neue Software für maximale Effizienz aus.
\\
So werden die Daten nicht mehr in mehreren Datenbanken gespeichert, sondern nur noch im In-Memory System. Die Daten wurden bei OLTP und OLAP Systemen meist auf Festplatten oder Flash-Speicher gespeichert. Da bei In-Memory Systemen die Daten jedoch im DRAM gespeichert werden, erhält man dadurch deutlich geringere Zugriffszeiten.
\\
Die Daten bei einem OLTP-System werden zeilenbasiert gespeichert, da man dadurch einfacher einzelne Tupel aus einer Datenbanktabelle lesen kann. Dies ist jedoch nicht sehr dafür geeignet, mehrere Einträge aus einer Spalte auszulesen.
In-Memory Systeme speichern deshalb ihre Daten spaltenbasiert.
\\
Der Vorteil bei Spalten basierten Speicherung ist, dass bestimmte Operationen, wie zum Beispiel die Aggregation schneller ausgeführt werden können. Bei einer zeilenbasierten Speicherung kann es passieren, dass man aus jeder Zeile nur ein Wert dafür braucht, jedoch jedes mal die gesamte Zeile einlesen muss. Bei einer spaltenbasierten Speicherung kann man jedoch einfach die gesamte Spalte dafür einlesen und muss keine unnötigen Daten in den CPU Cache laden.
Da man die gesamte Datenbank in Main Memory dauerhaft speichern will, sind die Daten auch komprimiert um diese möglichst klein zu halten.
Durch die spaltenbasierte Speicherung ist dies besonders gut möglich, da Daten von gleichen Typen jeweils zusammen gespeichert werden und es somit wahrscheinlich ist, das Werte mehrfach auftreten, da viele Spalten meist eine geringe Kardinalität Werten besitzen. Auch NULL Values oder Default Values sind meist der einzige Inhalt einer Spalte und somit komprimierbar. Dadurch kann man mit zum Beispiel einer Dictionary-Komprimierung viel Speicherplatz sparen.
Weiterhin nutzen In-Memory Systeme die Entwicklung von Multicore-CPUs aus. Die Prozesse werden dabei in kleinste Verarbeitungsschritte aufgeteilt, die jeweils dann auf mehreren Prozessoren parallel ausgeführt werden können.\\
All diese Veränderungen der Architektur sorgen dafür, dass man die Daten immer 
aktuell im In-Memory System hat. Analysen werden nun immer über 
den aktuellsten Daten durchgeführt und nicht über älteren, da die Daten noch nicht vom OLTP zum OLAP System kopiert wurden. Weiterhin sind durch die geringere Zugriffszeit auf die Daten, die Parallelisierung und die Speicherung der Daten in Spaltenformat die Geschwindigkeit von Prozessen deutlich erhöht worden. Somit ist es für Nutzer der In-Memory Systeme möglich, Analysen in Echtzeit zu berechnen und vor allem die Anforderungen für die Analysen variabel ändern und es muss nicht mehr auf die Ergebnisse gewartet werden. Dadurch erledigt sich das Vorberechnen von Ergebnissen.
\\

\newpage
%
\section{In-memory basiertes Rechnungswesen}
In der Zeit vor den In-Memory-Systemen gab es beim Rechnungswesen einer Firma Accounting Documents, in denen zum Beispiel die Belegköpfe und -körper enthalten sind, aber auch sekundäre Indizes. Weiterhin gab es bereits materialisierte Aggregate und Views, die aufwändige Queries schon zu einem Zeitpunkt vorberechnen, damit diese dann als Cache benutzt werden können. Außerdem kam die Change History, also die Änderungen zwischen dem OLTP- und dem OLAP-System, zum Einsatz. Dadurch, dass die Accounting Documents, also die Standardtabellen noch zusätzlich auf dem OLAP- und auf dem OLTP-System gespeichert sind, sind die Daten bereits doppelt verfügbar und müssen für die materialisierten Views und Aggregate noch öfter abspeichern.Es werden also viele redundante Datensätze gehalten, die jedoch nötig waren, um überhaupt Analysen auf den Daten in einer annehmbaren Zeit durchführen zu können.
\\
Weiterhin waren die Daten in dem Data Warehouse(OLAP-System) nicht aktuell, es kann also öfters passieren, dass man die Analyse durchführt und die jüngsten Daten in der Datenbank schon über einen Monat alt sind.
Bei einem In-Memory Datensystem sollen die Daten redundanzfrei gespeichert werden. Man soll also nur noch die Belegköpfe und -körper haben , also nur die Basistabellen und sonst nichts. Die Analyse und auch die Transaktionen sollen jedoch trotzdem deutlich effizienter ablaufen.
\\
Dadurch, dass man nur noch ein System hat, spart man sich natürlich auf lange Sicht die Kosten der mehreren Server(OLTP- und OLAP- Systeme) und braucht nur noch ein In-Memory System. Damit kann man langfristig schon an Anschaffungs- und Betriebskosten der Server sparen.
\\
Da man nur noch die Basistabellen speichert, kann man die Menge an Daten, die man sonst speichern müsste um mindestens die Hälfte verringern. Man braucht somit also nun nur noch die Hälfte an Speicherkapazität wie vorher.
\\
Weiterhin hat man nun alle Daten in nur zwei Tabellen und kann einfach so auf diese zugreifen und muss nicht mit materialisierten Views oder Aggregaten arbeiten. Die Queries werden dadurch also einfacher zu schreiben und zu verstehen. Da man die Anzahl möglicher Joins, die sehr zeitaufwendig sein können, reduziert, indem man nur noch die zwei Tabellen hat, ist es auch einfacher möglich effiziente Queries in kürzerer Zeit zu schreiben.
Da Queries also schneller laufen, siehe Aufgabe 3, und man schneller diese Queries auch effizient schreiben kann, ist es auch möglich in Echtzeit Analysen durchzuführen und diese auch in Echtzeiten zu ändern, so wie sie gerade benötigt werden.
\\
Die Queries müssen also nicht 3 Monate, bevor man etwas wissen will, materialisiert werden, damit man dann irgendwann ein Ergebnis bekommt, sondern man kann in Echtzeit die aktuellen Daten, die auch wirklich aktuell sind, da keine Trennung zwischen den OLAP und OLTP Daten mehr herrscht, analysieren und somit immer sofort auf mögliche Änderungen reagieren und sein Geschäft dementsprechend anpassen.Man kann somit mögliche Fehler verhindern oder aber auch mögliche Chancen ausnutzen. 
\\
Die Benutzung einer In-Memory Technologie kann also die Komplexität der Anwendungssoftawareschichten reduzieren, Durchführung von datenintensiven Aufgaben nahe der Datenquelle ermöglichen, die Reaktionszeiten beschleunigen und dadurch die Zeit des Benutzers sparen sowie die Verwaltungsaufgaben mit Massendaten beschleunigen.
Aus der betriebswirtschaftlichen Sicht kann ein In-Memory System dessen Benutzern helfen schnellere, informiertere Entscheidungen zu treffen und Reporting und Analyse-Anwendungen anzutreiben. Dieses hingegen kann die Business-Prozesse stärken, dass zu verbesserten Ergebnissen führen kann. Zum Beispiel die Umstellung von wöchentlicher auf eine stündliche Umsatzprognose würde die Erstellung von Echtzeit- und Produktpreismodellen ermöglichen, dass weiterhin die Rentabilität steigern könnte. Dieses erfordert natürlich auch, dass Entscheidungen schnell kommuniziert und ausgeführt werden können.
\cite{In-memory}

\newpage
\section{Reflexion des Projektes}
Vor und Nachteile von Play! / Marmolata bei Implementierung von Enterprise Applications
Für welche Aspekte ergibt die Nutzung von Frameworks Sinn und wo ist Herausforderungen

Herkömmliche Programmiermodelle haben die Vorteil, dass man mit ihnen alle möglichen Designs erstellen kann. Enterprise Frameworks haben da meist den Nachteil, dass sie auf den Style einer bestimmten Firma festgelegt sind und Anpassungen sehr aufwendig sind, falls man versucht Dinge zu realisieren, die gegen die Designstandards des Unternehmens verstoßen. Dadurch, dass die Designs so fest in das Framework integriert sind, hat jede mit dem Framework programmierte Anwendung natürlich einen gewissen Stil, der Wiedererkennungswert hat. Dadurch erreicht man zum Beispiel, dass alle Anwendungen, die in einem Unternehmen programmiert haben, direkt diesem Unternehmen zugeordnet werden können. 
Genauso wie es gut ist, dass herkömmliche Programmiermodelle viel Freiheit bei dem Design und der Funktionalität bieten, wird es ihnen auch zum Nachteil. In den meisten Fällen sind sie nämlich nicht auf einen Enterprise Kontext optimiert und es erfordert viel Arbeit alle nötigen Standards und Usability-Features zu implementieren. Das ist natürlich bei Enterprise Frameworks nicht der Fall.

\cite{example}


%Hier kommt das Literaturverzeichnis
\newpage

\addcontentsline{toc}{section}{Literaturverzeichnis} % Zeile für das Inhaltsverzeichnis

\bibliography{bibfile}
\bibliographystyle{alphadin}

\end{document}
