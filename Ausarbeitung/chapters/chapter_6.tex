%
\section{Der Abschluss der Seminararbeit}
\label{sec_conclusion}
In diesem Kapitel wird noch einmal zusammengefasst, wie das Schlusskapitel einer wissenschaftlichen Arbeit aufgebaut werden sollte und welche wichtigen Punkte darin enthalten sein sollten.
Den Abschluss des Kapitels bietet eine kleine Hilfestellung bezüglich der allgemeinen Anforderungen an das Schreiben einer wissenschaftlichen Arbeit.

\subsection{Zusammenfassung und Ausblick}
Das letzte Kapitel "`Zusammenfassung und Ausblick"' soll die gewonnenen Ergebnisse und Erkenntnisse Ihrer Arbeit knapp zusammenzufassen.
Stellen Sie dabei eindeutig klar, was wichtig ist und was nicht.
Dazu zählt auch, dass Sie einen Ausblick auf die Weiterentwicklung innerhalb des von Ihnen bearbeiteten Themengebiets geben können.
\begin{itemize}
\item Was haben Sie erreicht?
\item Welche Schlussfolgerungen lassen sich aus Ihren Ergebnissen zusammenfassend ziehen?
\item Was sind die nächsten Schritte?
\item Wie kann der vorgestellte Ansatz weiter verwendet werden?
\end{itemize}

Der Abschluss des vorliegenden Dokuments zum Schreiben einer wissenschaftlichen Arbeit könnte beispielsweise folgendermaßen erfolgen:

\smallskip

Die vorliegende Arbeit fasst die wichtigsten Schritte zum Verfassen einer wissenschaftlichen Arbeit im Fachbereich Informatik zusammen und zielt auf den Einsatz im universitären Umfeld zum Verfassen von Studien-, Seminar-, und Abschlussarbeiten ab. 
Aufbauend auf die in der vorliegenden Arbeit erläuterten Arbeitsweisen sind die Studenten in der Lage, selbstständig eine wissenschaftliche Arbeit zu verfassen.
Dabei wird auf den allgemeinen Aufbau der schriftlichen Ausarbeitung sowie auf die technische Umsetzung mit Hilfe des in der Informatik oft eingesetzten Satzsystems \LaTeX \ bzw. BibTeX näher eingegangen.  
Die Arbeit sollte durch eine Präsenzlehrveranstaltung für die Studierenden unterstützt werden, um entstandene Fragen direkt beantworten zu können und um zusätzliche, aktuelle Hinweise zu geben.


\subsection{Anforderungen an das Schreiben einer wissenschaftlichen Arbeit}
Es gibt zahlreiche Hilfestellungen zum Schreiben wissenschaftlicher Arbeiten, die sich natürlich oft nach dem jeweiligen Fachbereich, in dem eine wissenschaftliche Arbeit erstellt werden soll unterscheiden \cite{Zobel:2004:WCS:993467, Evans:2014:WBT:2633585}.
Eine der bekanntesten stammt von Umberto Eco, dem berühmten Autor des Buchs \glqq \textit{Der Name der Rose}\grqq \ und Professor für Semiotik, der seine Anleitung zum Schreiben einer wissenschaftlichen Arbeit als anschaulich zu lesendes Buch herausgegeben hat\footnote{Allerdings sind seine praktischen Ratschläge manchmal mit Vorsicht zu genießen, da sie auf den italienischen Universitätsbetrieb abzielen.}.

\smallskip

Umberto Eco fasst die Anforderungen an das Schreiben einer wissenschaftlichen Arbeit in den folgenden vier Kriterien zusammen~\cite{Eco88}:
\begin{enumerate}
\item \glqq Die Untersuchung behandelt einen erkennbaren Gegenstand, der so genau umrissen ist, dass er auch für Dritte erkennbar ist.\grqq
\item \glqq  Die Untersuchung muss über diesen Gegenstand Dinge sagen, die noch nicht gesagt worden sind, oder sie muss Dinge, die schon gesagt worden sind, aus einem neuen Blickwinkel sehen.\grqq
\item \glqq Die Untersuchung muss für andere von Nutzen sein.\grqq
\item \glqq Die Untersuchung muss jene Angaben enthalten, die es ermöglichen nachzuprüfen, ob ihre Hypothesen falsch oder richtig sind, sie muss also die Angaben enthalten, die es ermöglichen, die Auseinandersetzung in der wissenschaftlichen Öffentlichkeit fortzusetzen.\grqq
\end{enumerate}