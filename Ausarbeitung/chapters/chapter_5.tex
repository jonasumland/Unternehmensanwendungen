\section{Reflexion des Projektes}
Vor und Nachteile von Play! / Marmolata bei Implementierung von Enterprise Applications
Für welche Aspekte ergibt die Nutzung von Frameworks Sinn und wo ist Herausforderungen

Herkömmliche Programmiermodelle haben die Vorteil, dass man mit ihnen alle möglichen Designs erstellen kann. Enterprise Frameworks haben da meist den Nachteil, dass sie auf den Style einer bestimmten Firma festgelegt sind und Anpassungen sehr aufwendig sind, falls man versucht Dinge zu realisieren, die gegen die Designstandards des Unternehmens verstoßen. Dadurch, dass die Designs so fest in das Framework integriert sind, hat jede mit dem Framework programmierte Anwendung natürlich einen gewissen Stil, der Wiedererkennungswert hat. Dadurch erreicht man zum Beispiel, dass alle Anwendungen, die in einem Unternehmen programmiert haben, direkt diesem Unternehmen zugeordnet werden können. 
Genauso wie es gut ist, dass herkömmliche Programmiermodelle viel Freiheit bei dem Design und der Funktionalität bieten, wird es ihnen auch zum Nachteil. In den meisten Fällen sind sie nämlich nicht auf einen Enterprise Kontext optimiert und es erfordert viel Arbeit alle nötigen Standards und Usability-Features zu implementieren. Das ist natürlich bei Enterprise Frameworks nicht der Fall.

\cite{example}