\section{Reflexion des Projektes}
Herkömmliche Programmiermodelle haben den Vorteil, dass man mit ihnen alle möglichen Designs erstellen kann. Enterprise Frameworks haben da meist den Nachteil, dass sie auf den Style einer bestimmten Firma festgelegt sind und Anpassungen sehr aufwendig sind, falls man versucht Dinge zu realisieren, die gegen die Designstandards des Unternehmens verstoßen. Dadurch, dass die Designs so fest in das Framework integriert sind, hat jede mit dem Framework programmierte Anwendung natürlich einen gewissen Stil, der Wiedererkennungswert hat. Dadurch erreicht man zum Beispiel, dass alle Anwendungen, die in einem Unternehmen programmiert haben, direkt diesem Unternehmen zugeordnet werden können. 
\\
Ein weiterer Vorteil ist es, dass durch das Framework ein unternehmensweiter Sicherheitsstandard gewährleistet werden kann. Die einzelnen Anwendungsteams müssen sich nicht mit der Frage der Sicherheit beschäftigen, da diese Funktion durch das Enterprise-Framework abgenommen wird. So mit kann sichergestellt werden, dass sensible Daten unabhängig von der jeweiligen Anwendung immer angemessen geschützt werden.
\\
Genauso wie es gut ist, dass herkömmliche Programmiermodelle viel Freiheit bei dem Design und der Funktionalität bieten, wird es ihnen auch zum Nachteil. In den meisten Fällen sind sie nämlich nicht auf einen Enterprise-Kontext optimiert und es erfordert viel Arbeit alle nötigen Standards und Usability-Features zu implementieren. Das ist bei Enterprise Frameworks nicht der Fall. Als Beispiel aus den praktischen Übungen sind hier die Formatierung von Zahlen, Daten und Währung anzuführen. Ebenfalls war die Datenbeschaffung aufwändiger.
\\
Neben Designanpassungen haben herkömmliche Programmiermodelle den Vorteil, dass man sich allen bisher veröffentlichten Frameworks bedienen kann und je nach Anforderung integrieren kann. So konnten wir in unsere Übung zu herkömmlichen Programmiermodellen einfach Charts und Grafiken hinzufügen. Durch die hohe Spezialisierung des Enterprise-Frameworks war es nicht möglich, sich anderen Bibliotheken zu bedienen und Graphen hinzu zu fügen.
\\
Im Allgemeinen lässt sich sagen für die Grundfunktionalitäten, die man benötigt, wenn man Enterprise Applications verwendet, sind entsprechende Frameworks gut und leicht zu verwenden. Allerdings sollte sich ein Enterprise Framework in herkömmliche Programmiermodelle integrieren lassen, um Programmierern die Möglichkeit ihr Lieblingslayoutframework weiter verwenden zu können und trotzdem die Usability und Designstandards von Enterprise Applications einhalten zu können. Es ist einfach nicht realistisch, dass ein Enterpriseframework alle Funktionalitäten und Layoutoptionen enthält, die ein moderner Webdeveloper benötigt.