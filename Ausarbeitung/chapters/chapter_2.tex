%
\section{Aufgaben des externen und des internen Rechnungswesens}

\subsection{Aufgaben des internen Rechnungswesen}
Das interne Rechnungswesen wird auch als Controlling bezeichnet. Im Controlling wird durch Planung, Kontrolle und Koordination die Effizienz bewerteter Prozesse des Unternehmens verbessert und somit der Unternehmenserfolg maximiert. Dazu werden vor allem die Kosten- und Leistungsrechnung als auch die Investitionsrechnung herangezogen.
\\
Das interne Rechnungswesen richtet sich ausschließlich an Unternehmensinterne, unterliegt in der Regel keinen gesetzlichen Bestimmungen und ist daher optional. Je nach konkretem Rechnungszweck sind die angewandten Zeitintervalle unterschiedlich. So wird für die Investitionsrechnung ein mehrjähriger Zeitraum festgelegt, während die Kosten- und Leistungsrechnung mehrmal pro Jahr erfolgt. Hinzu kommt, dass verschiedene Unternehmensbereiche getrennt voneinander betrachtet werden können.
\\
Neben der Wirtschaftlichkeit muss das Controlling allerdings auch darum kümmern das Unternehmen mit Informationen über der Markt, Marktlücken und Trends zu analysieren und aufzubereiten. Mit all diesen Informationen soll dem Management ermöglicht werden fundierte Entscheidungen über die Zukunft des Unternehmens zu treffen.
Die Kernbereiche des internen Rechnungswesens sind Kostenrechnung, Finanzplanung und -steuerung, sowie Leistungsmessung. 
\\
Kostenrechnung schafft eine Grundlage für die wirtschaftliche Entscheidungsfindung, Wirtschaftlichkeitsberechnung und die Berichterstattung der Organisation. Finanzplanung und Beratung, wiederum quantifizieren Pläne für die Organisation, setzen Ziele für Personal und Abteilungen und folgen die Erreichung der Ziele. Mit Hilfe der Steuerung werden die Strategien von der Organisation in die Praxis umgesetzt und deren Umsetzung überwacht. Einstellen von strategischen Zielen und die Verwendung von nicht-finanziellen Indikatoren hat ebenfalls eine wichtige Rolle im internen Rechnungswesen.\cite{madegowda2006}

\subsection{Aufgaben des externen Rechnungswesen}
Das externe Rechnungswesen bildet die finanzielle Situation des Unternehmens nach außen ab und richtet sich an Personen/Institutionen, denen ein berechtigtes Interesse an Informationen über das Unternehmen zugeschrieben wird. Dies beinhaltet Kapitalgeber, Gläubiger, das Finanzamt, Kunden und Lieferanten, den Staat und auch die interessierte Öffentlichkeit. Das Ziel des externen Rechnungswesen ist eine vergangenheitsorientierte Dokumentation, sowie die Rechenlegung. 
\\
Im Gegensatz zum internen Rechnungswesen ist das externe Rechnungswesen nicht optional und erfolgt in Form des Jahresabschlusses. In Deutschland richtet sich das externe Rechnungswesen daher nach den im HGB festgelegten gesetzlichen Bestimmungen. Für Personenunternehmen und Einzelkaufleute beinhaltete der Jahresabschluss die Bilanz und die Gewinn- und Verlustrechnung. Bei Kapitalgesellschaften kommt ein Lagebericht, der eine Darstellung der Unternehmensführung ist, sowie ein Anhang mit zusätzlichen Informationen zu Bilanz und Gewinn- und Verlustrechnung hinzu. Bestimmte Kapitalgesellschaften müssen außerdem eine Kapitalflussrechnung und einen Eigenkapitalspiegel hinzufügen.\cite{boeckler}
\\
Zu den Aufgaben des externen Rechnungswesen gehört auch die Dokumentation und Durchführung von Geschäftsvorfällen. Bei der Dokumentation müssen sowohl logistische als auch nicht-logistische Geschäftsvorfälle berücksichtigt werden. Die Durchführung von Geschäftsvorfällen beinhaltet logischer Weise nicht das Kerngeschäft sondern deren Fortführung. Das beinhaltet Zahlen und Mahnen. Nicht-logistische Geschäftsvorfälle werden vollständig durchgeführt. Nicht-logistische Vorfälle sind zum Beispiel Bargeschäfte und Finanzierung.

\subsection{Kontenmodell}
Im Kontenmodell werden Geschäftsvorfälle als Finanzbuchhaltungsbelege beschrieben. Dieser Finanzbuchhaltungsbeleg verweist auf den logistischen Beleg. Die in dem Beleg beschriebenen Positionen werden in Form von Konten beschrieben. Ein Geschäftsvorfall wird nun in eine von zwei Klassen eingeordnet. Ist es eine werterhöhende Position, ist es in der Klasse der SOLL-positionen, ist es eine werterniedrigende Position, in der Klasse er HABEN-positionen.

\subsection{Treibermodell}
Im Treibermodell werden Geschäftsvorfälle als Ursache-Wirkungsbeziehungen abgebildet. Als Wirkung auslösende Treiber werden in diesem Modell die logistischen Geschäftsvorfälle verwendet. Da es aber auch andere Gründe für Profitabilitätänderung geben kann, können auch beliebige sonstige Ereignisse der realen Welt ein Treiber sein.
Die Werte aus denen sich die Profitabilität ergibt, werden aus den Treibern errechnet. Die Formeln dafür können sich aus Standards, Erfahrungswerten aber auch rechtlichen Vereinbarungen ergeben.
\\
Die Werte aus denen sich die Profitabilität ergibt, werden aus den Treibern errechnet. Die Formeln dafür können sich aus Standards, Erfahrungswerten aber auch rechtlichen Vereinbarungen ergeben. 


\subsection{Vergleich}
Durch die Dokumentation der Vorfälle in SOLL und HABEN wird die Realität direkt abgebildet. Darüber hinaus kann der Bezug zu logistischen Objekten ohne Probleme hergestellt werden. Außerdem wird von dem konkreten logistischen Objekt abstrahiert und vereinheitlicht. Dadurch hat man die Möglichkeit konkrete Objekte, auch ohne ihren Kontext zu analysieren, betrachten zu können.
\\
Es ist keine direkte Auswirkung von einem Vorfall auf die Gesamtprofitabilität des Unternehmens zu bemerken. Um doch Informationen über Zusammenhänge von Vorfällen auf die Wirtschaftlichkeit zu erhalten, müssen nun zusätzliche Modelle erstellt werden.
\\
Ein Treibermodell ermöglicht eine wirksame Kontrolle mit Hilfe Soll-Ist-Vergleichen und eine Entscheidungsunterstützung in Form von Simulationen. Somit können sowohl die Auswirkungen als auch die Ursachen für Vorfälle ermittelt und ausgewertet werden.
\\
Wenn die Einfluss stärksten Treiber ermittelt wurden, sorgt das für eine Komplexitätsreduktion in der Planung und zur Fokussierung auf die wesentlichen Performancetreiber im Unternehmen. Außerdem kann durch die stärkere Top-Down-Orientierung und enge Verschränkung der operativen Planungszyklen mit der Strategie (strategische „Vorsteuerung“ des finanziellen Erfolgs) leichter und effizienter Entscheidungen getroffen werden. Durch das Aufschlüsseln der Informationen erhält der Analyseprozess mehr Transparenz hinsichtlich der Performancetreiber in den dezentralen Unternehmenseinheiten. Dadurch entsteht eine bessere Nachvollziehbarkeit von Planungsprämissen und Gründen für Planabweichungen.
\\
Falls dann doch mal etwas schief läuft, haben die Top-Manager einen unmittelbaren Andockpunkt für Sensitivitätsanalysen („Stress-Tests“) und Simulationsmodelle.\cite{raoul}
\\
Um wirklich den Grund für eine Einbuße in der Profitabilität zu bemerken, müssen alle Faktoren, die einen Einfluss haben erfasst worden sein. Ist dies nicht der Fall, bietet das Treibermodell unter Umständen falsche oder unvollständige Aussagen.
Außerdem ist das Erstellen eines solchen Modells sehr aufwändig. Wenn man darüber hinaus noch von der Reduktion der Einflussfaktoren gebrauch machen möchte erfordert dies extra Aufwand.


