%
\section{Aufbau und Inhalt der Seminararbeit}
\label{sec_aufbau}

Im vorangegangenen Kapitel hatten wir bereits die Gliederung einer Seminararbeit kurz vorgestellt und erläutert, welche inhaltlichen Punkte in der \glqq Einleitung\grqq\, behandelt werden sollten.
Die folgenden Abschnitte skizzieren inhaltlich die übrigen der bereits genannten Gliederungspunkte.

\smallskip

Wie Sie sehen, ist es wichtig, Überleitungen zwischen den einzelnen Kapiteln anzulegen.
Überleitungen erleichtern es dem Leser, den Überblick über das Thema zu behalten und das jetzt folgende Kapitel richtig in den Gesamtzusammenhang einzuordnen.
Enthält das Kapitel mehrere Unterkapitel, empfiehlt es sich diese hier an dieser Stelle kurz zusammenzufassen.

\smallskip

Für die vorliegende Musterausarbeitung könnte diese Zusammenfassen etwa folgendermaßen aussehen:
Zunächst wird auf die Bedeutung verwandter Arbeiten und des wissenschaftlichen Hintergrunds eingegangen, bevor anschließend der eigene Lösungsansatz für die gestellte Aufgabe vorgestellt wird. 
Danach folgen einige Tipps, was bei der Darstellung der Evaluation als Beleg für die Gültigkeit und die Qualität der gewählten Problemlösung zu beachten ist, gefolgt von Hilfestellungen für die Diskussion der erzielten Ergebnisse.
Im letzten Punkt wird auf die Bedeutung der Zusammenfassung und des Ausblicks auf anschließende Forschungsarbeiten eingegangen.


\subsection{Verwandte Arbeiten und wissenschaftlicher Hintergrund (Related Work)}
%%
Hier sind vor allem zwei inhaltliche Punkte zu berücksichtigen:
\begin{itemize}
\item {\bf Vorarbeiten und Grundlagen, die zum Verständnis der Arbeit notwendig sind}\\
Keine bzw. kaum eine Arbeit beginnt als \glqq tabula rasa\grqq , d.h. meist bauen wir auf  vorhandenen Grundlagen bzw. Vorarbeiten auf.
Die zum Verständnis der eigenen Arbeit notwendigen wissenschaftlichen Grundlagen und Voraussetzungen müssen in diesem Kapitel skizziert bzw. zusammengefasst werden.
Dabei sollte man vom durchschnittlichen Kenntnisstand eines Informatikers ausgehen, d.h. Allgemeinplätze und allzu Grundlegendes hat hier nichts zu suchen.
Genauso soll hier nicht notwendigerweise eine kompletter Wissenschaftszweig in epischer Tiefe ausgebreitet werden, sondern lediglich die zum Verständnis notwendigen Teilbereiche in skizzenhafter Form und mit Angabe von weiterführenden Literaturhinweisen zusammengefasst werden. 

\smallskip

Zum Beispiel könnten an dieser Stelle die Grundlagen und Vorzüge von Linked Open Data und deren Bezug zur vorliegenden Arbeit erläutert werden.

\smallskip

\item {\bf Alternative Ansätze und Forschungsarbeiten zum Thema}\\
Besonders wichtig ist es, spezielle Vorarbeiten und alternative Lösungsansätze zum behandelten Thema darzulegen.
Gibt es zu der von Ihnen gewählten Problemstellung bereits alternative Lösungen, die einen anderen oder vergleichbaren Ansatz verfolgen? Wie unterscheiden sich diese Lösungen von Ihrem eigenen Ansatz, wo liegen Vor- und Nachteile des jeweiligen Ansatzes? Grenzen Sie ihren eigenen Forschungsansatz von den alternativen referenzierten Ansätzen argumentativ ab.

\smallskip

So könnten Sie erwähnen, dass Ihr eigener Ansatz z.B. von den selben Voraussetzungen ausgeht wie Ansatz XY, im Gegensatz zu diesem aber auf dem Einsatz von Linked Data Technologien beruht, und daher bessere / genauere / umfangreichere Ergebnisse erzielt. (Bitte wiederholen Sie diese Argumentation nicht. Sie dient nur als illustrierendes Beispiel.)

Wichtig ist, dass Sie jede der vorgestellten, alternativen wissenschaftlichen Arbeiten 
\begin{itemize}
\item korrekt zitieren (Bibliografie),
\item kurz die wichtigsten Ergebnisse bzw. Strategien skizzieren und
\item diese (kurz und knapp) in Zusammenhang mit ihrer eigenen Arbeit stellen. 
\end{itemize}
Wie unterscheidet sich der eigene Ansatz von den vorgestellten Arbeiten? 
Warum ist der eigene Ansatz eventuell erfolgsversprechender? 

\end{itemize}

\subsection{Eigener Ansatz zur Lösung der gestellten Aufgabe (Method and Approach)}
%%
Hier haben Sie die Freiheit, Ihren eigenen Forschungsarbeiten angemessen viel Raum zur Verfügung zu stellen.
Achten Sie dabei auf einen logischen Aufbau der Darstellung, d.h. Grundlegendes stets zuerst.
\begin{itemize}
\item Wie sind Sie vorgegangen?
\item Wo gab es (welche) Probleme?
\item Wie wurden diese gelöst?
\item Schreiben Sie in verständlicher Weise und drücken Sie sich dabei jeweils möglichst präzise, d.h. unmissverständlich aus (vgl. Kap.~\ref{sec_stil})
\item Verwenden Sie Abbildungen, Tabellen und Beispiele, um Ihren Ansatz besser zu erläutern. Stellen Sie ein Softwaresystem vor, dann hilft eine Architekturskizze. Stellen Sie einen Workflow vor, dann hilft ein Workflow-Diagramm.
\item Setzen Sie kein Wissen als implizit vorhanden voraus, sondern sprechen Sie explizit alle Probleme und wichtigen Fakten an.
\item Wichtig: Was Sie hier nicht beschreiben, können wir nicht bewerten!
\end{itemize}
Bedenken Sie dabei stets, dass ein Leser nicht dasselbe Wissen besitzen kann wie Sie und dass Sie ihm deshalb alle ihre Ergebnisse erklären müssen.

\smallskip

Verwenden Sie bei der Darstellung Ihres Lösungsansatzes eine möglichst einfache Sprache.
Vermeiden Sie zahlreiche Schachtelungen und Nebensätze. Der dargestellte Sachverhalt ist meist bereits hinreichend komplex. 
Die zu seiner Darstellung verwendete Sprache sollte sollte den Zugang für den Leser nicht auch noch zusätzlich erschweren.


\subsection{Evaluation des wissenschaftlichen Beitrags (Evaluation)}
%
Viele wissenschaftliche Aufgabenstellungen erfordern den Nachweis der Qualität bzw. der Effizienz des vorgestellten Lösungsansatzes, d.h. das entwickelte Verfahren bzw. die vorgestellte technische Lösung muss objektiv getestet und anschließend bewertet werden.
Man unterscheidet hier grundsätzlich zwischen quantitativer und qualitativer Evaluation.
Während in der qualitativen Evaluation oft menschliche Testpersonen die Qualität einer Lösung entsprechend vorgegebener Qualitätskriterien beurteilen müssen, stützt sich die quantitative Evaluation auf vorgegebene Testdatensätze und einen Soll-/Ist-Vergleich.
Testdatensätze (auch Benchmarks) enthalten korrekte, oft manuell verifizierte Ergebnisse zu vorgegebenen Aufgabenstellungen (Ground Truth).
Diese werden quantitativ mit den Ergebnissen des von Ihnen implementierten Verfahrens verglichen und in Form von statistischen Kenngrößen (z.B. Recall, Precision und F$_1$-Measure) beschrieben.

\begin{itemize}
\item Verwenden Sie (wenn möglich) vorgegebene Standard Benchmarks, um eine möglichst breite Vergleichbarkeit mit alternativen Lösungsansätzen bzw. Systemen zu ermöglichen und die in der Wissenschaft notwendige Nachvollziehbarkeit zu gewährleisten.
\item Wenn Sie eigene Testdaten bzw. Benchmarks zusammenstellen, stellen Sie diese öffentlich (im WWW) zur Verfügung, um zukünftig bessere Vergleichberkeit und Nachvollziehbarbeit zu gewährleisten.
\end{itemize}

%\bigskip

Nicht jede Aufgabenstellung ist für eine solche Evaluation geeignet. Theoretische und mathematische Aufgabenstellungen resultieren oft in Lösungen, deren Gültigkeit mit mathematischen Methoden (mathematischen Beweisverfahren) belegt werden muss.



\subsection{Diskussion der erzielten Ergebnisse (Discussion)}
%%
In diesem Kapitel sollten Sie Ihre erzielten Ergebnisse (möglichst vollständig) präsentieren und deren Qualität diskutieren.
Sollte der zur Verfügung stehende Platz für eine vollständige Darstellung im Rahmen der Arbeit nicht ausreichen, stellen Sie an dieser Stelle die wichtigsten Resultate dar und verweisen Sie auf eine vollständige Version, die Sie z.B. auf einer entsprechenden Webpage bereithalten. 
Dabei sollten (falls jeweils zutreffend) folgende Fragen beantwortet werden:
\begin{itemize}
\item Was wurde erreicht, was kann noch verbessert werden bzw. wo gibt es noch offene (evtl. aus Zeitgründen nicht implementierte) Punkte?
\item Sind die erzielten Ergebnisse objektiv? Gibt es Gründe, daran zu zweifeln?
\item Warum ist der eigene Ansatz besser/schlechter als die zum Vergleich herangezogenen?
\item Welche Vorbedingungen könnten verändert werden, um eventuell bessere Ergebnisse zu erzielen?
\item Wenn die Evaluation nicht aussagekräftig genug ist, wie könnte man sie noch verbessern?
\item Was haben Sie aus dem Seminar mitgenommen (z.B. Wo liegen die Vorteile der im Seminar behandelten wissenschaftlichen Methoden und Verfahren?)
\end{itemize}
Der Leser sollte nach diesem Kapitel davon überzeugt sein, dass Sie Ihre Arbeit gut gemacht haben und alle offensichtlich aufgetretenen Fragen nach bestem Wissen und Gewissen beantwortet haben.

\subsection{Zusammenfassung und Ausblick (Conclusion and Outlook)}

In diesem Abschnitt werden die erzielten Ergebnisse noch einmal kurz zusammengefasst und es wird ein kurzer Ausblick auf weiterführende Forschungs- und Entwicklungsarbeiten gegeben, die sich im Zuge Ihrer Arbeiten ergeben haben (vgl. Kap.~\ref{sec_conclusion}). Hier können Sie z.B. ausführen, welche Arbeiten Sie aus Zeitgründen nicht mehr umsetzen konnten, aber für wichtig oder sinnvoll erachten. 
Unterscheiden Sie, welche Verbesserungsmaßnahmen kurzfristig bzw. langfristig vorgenommen werden könnten und versuchen Sie den dadurch erzielten Gewinn bzw. Vorteil zu quantifizieren.