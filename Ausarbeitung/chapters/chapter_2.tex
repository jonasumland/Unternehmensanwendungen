%
\section{Aufgaben des externen und des internen Rechnungswesens}

\subsection{Aufgaben des internen Rechnungswesen}
Das interne Rechnungswesen wird auch als Controlling bezeichnet. Im Controlling wird durch Planung, Kontrolle und Koordination die Effizienz bewerteter Prozesse des Unternehmens der Unternehmenserfolg maximiert. Dazu werden vor allem die Kosten- und Leistungsrechnung als auch die Investitionsrechnung herangezogen.
Neben der Wirtschaftlichkeit muss das Controlling allerdings auch darum kümmern das Unternehmen mit Informationen über der Markt, Marktlücken und Trends zu analysieren und aufzubereiten. Mit all diesen Informationen soll dem Management ermöglicht werden fundierte Entscheidungen über die Zukunft des Unternehmens zu treffen.

\subsection{Aufgaben des externen Rechnungswesen}
Im externen Rechnungswesen geht es darum die finanzielle Situation des Unternehmens nach außen abzubilden. Dies beinhaltet Kapitalgeber, Gläubiger, das Finanzamt, Kunden und Lieferanten, den Staat und auch die interessierte Öffentlichkeit.
Im Gegensatz zum internen Rechnungswesen ist das externe Rechnungswesen nicht optional und erfolgt in Form des Jahresabschlusses. Der Jahresabschluss beinhaltet die Bilanz als auch die Gewinn- und Verlustrechnung.
Zu den Aufgaben des externen Rechnungswesen gehört auch die Dokumentation und Durchführung von Geschäftsvorfällen. Bei der Dokumentation müssen sowohl logistische als auch nich-logistische Geschäftsvorfälle berücksichtigt werden. Die Durchführung von Geschäftsvorfällen beinhaltet logischer Weise nicht das Kerngeschäft sondern deren Fortführung. Das beinhaltet Zahlen und Mahnen. Nicht-logistische Geschäftsvorfälle werden vollständig durchgeführt. Nicht-logistische Vorfälle sind zum Beispiel Bargeschäfte und Finanzierung.

\subsection{Kontenmodell}
Allgemein:
Im Kontenmodell werden Geschäftsvorfälle als Finanzbuchhaltungsbelege beschrieben. Dieser Finanzbuchhaltungsbeleg verweist auf den logistischen Beleg. Die in dem Beleg beschriebenen Positionen werden in Form von Konten beschrieben. Ein Geschäftsvorfall wir nun in eine von zwei Klassen eingeordnet. Ist es eine werterhöhende Position, ist es in der Klasse der SOLL-positionen, ist es eine werterniedrigende Position, in der Klasse er HABEN-positionen.

\subsection{Treibermodell}
Allgemein:
Im Treibermodell werden Geschäftsvorfälle als Ursache-Wirkungsbeziehungen abgebildet. Als Wirkung auslösende Treiber werden in diesem Modell die logistischen Geschäftsvorfälle verwendet. Da es aber auch andere Gründe für Profitabilitätänderung geben kann, können auch beliebige sonstige Ereignisse der realen Welt ein Treiber sein.
Die Werte aus denen sich die Profitabilität ergibt, werden aus den Treibern errechnet. Die Formeln dafür können sich aus Standards, Erfahrungswerten aber auch rechtlichen Vereinbarungen ergeben.

\subsection{Vergleich}
KONTENMODELL
Vorteile:
Bildet die Realität ab und Bezug zu logistischen Objekten kann ohne Probleme hergestellt werden. Außerdem wird von dem konkreten logistischen Objekt abstrahiert und vereinheitlicht. Dadurch hat man die Möglichkeit konkrete Objekte auch ohne ihren Kontext zu analysieren, betrachten zu können.
\\
Nachteile:
Es ist keine direkte Auswirkung von einem Vorfall auf die Gesamtprofitabilität des Unternehmens zu bemerken.
\\
TREIBERMODELL
Vorteile:
Erlaubt Gründe für bestimmte Vorfälle zurück zu verfolgen.
\\
Nachteile:
Um wirklich den Grund für eine Einbuße in der Profitabilität zu bemerken, müssen alle Faktoren, die einen Einfluss haben erfasst worden sein. Ist dies nicht der Fall, bietet das Treibermodell unter Umständen falsche oder unvollständige Aussagen.

