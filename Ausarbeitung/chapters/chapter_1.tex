% Vierte Seite = Hier geht's eigentlich richtig los
\section{Grundbegriffe des Rechnungswesens}

\subsection{Balance Sheet (Bilanz)}
In der Bilanz wird die Vermögenslage dargestellt. Hierbei gibt es zwei Dinge zu unterscheiden.
Erstens: Woher kommt das Geld? Diese Mittel bezeichnet man als Passiva. Es besteht die Möglichkeit, dass das Unternehmensvermögen aus Eigen- oder auch aus Fremdkapital kommt. Auch fallen Rückstellungen und Verbindlichkeiten in die Kategorie der Passiva. Rückstellungen können zum Beispiel Steuerrückstellungen oder Pensionsrückstellungen sein. Verbindlichkeiten können gegenüber Anlegern, Geldinstituten oder Privatpersonen auftreten.
Zweitens: Aktiva beschreiben, was mit dem Geld gemacht wird. Geld kann entweder angelegt werden oder sich im Umlauf befinden. So gibt es bei den Aktiva einmal das Anlagevermögen und zum Anderen das Umlaufvermögen. Vermögen lässt sich in Immaterielle Dinge oder Sachen anlegen. Es ist auch möglich nur das Geld “arbeiten” zu lassen. Diese Form der Anlage bezeichnet man dann als Finanzanlage. Umlaufvermögen wird aufgeschlüsselt in Vorräte, Forderungen, Wertpapiere und Liquide Mittel.
Durch diese Aufschlüsselung ist es nun möglich zu sehen ob das Unternehmen Gewinn oder Verlust macht. Außerdem ist die Bilanz nötig um erkennen zu können woher das Geld kommt und wohin es investiert wird.

\subsection{Profit \& Loss Statement, P\& L (Gewinn- und Verlustrechnung)}
In der Gewinn- und Verlustrechnung geht es darum Einnahmen und Ausgaben aufzurechnen, damit man erfahren kann mit welchem Aufwand welcher Ertrag erreicht werden konnte. Dazu gibt es zwei verschiedene Verfahren. Das Erste ist das Umsatzkostenverfahren. Bei diesem Verfahren wird von den Erlösen die Kosten des Umsatzes und die Kosten der Funktionsbereiche abgezogen.
Bei dem Gesamtkostenverfahren verrechnet man den Erlös mit Bestandsänderungen und zieht dann die gesamten Kosten ab

\subsection{Cash Flow Statement (Kapitalflußtechnung)}
Jetzt bei dem Cash Flow Statement werden zusätzlich zu den Einnahmen und Ausgaben, die bei der P\& L-rechnung berücksichtigt werden zusätzlich noch Aufwände, die keine Auszahlungen sind, Auszahlungen, die keine Aufwände sind, Erträge, die keine Einzahlungen sind und Einzahlungen, die keine Erträge sind berücksichtigt. In diese Kategorien fallen zum Beispiel Abschreibungen, Investitionen und Desinvestitionen.
Das Cash Flow Statement erlaubt es Aussagen über die Fähigkeit des Unternehmens zum bezahlen seiner Rechnungen zu treffen.
