\section{Grundbegriffe des Rechnungswesens}

\subsection{Balance Sheet (Bilanz)}
Die Bilanz ist ein Teil des Abschlusses, der den ökonomischen Stand eines Unternehmens beschreibt. Im Vergleich zur Gewinn- und Verlustrechnung, die die Erträge und Aufwendungen eines bestimmten Zeitraumes darstellt, bezeichnet die Bilanz den Wert der Schulden und Vermögen einer Unternehmung zu einem bestimmten Zeitpunkt, wobei Aktiva und Passiva gegenübergestellt werden.\cite{businesson}
\\
Die Aktiva, auch Vermögen genannt, beschreiben hierbei den konkreten Verwendungszweck der eingesetzten Mittel. Posten der Aktiva sind das Anlagevermögen und das Umlaufvermögen. Das Anlagevermögen stellt langfristig an das Unternehmen gebundene Güter, die dem Geschäftsbetrieb dienen, wie zum Beispiel Grundstücke oder Maschinen dar. Das Umlaufvermögen beinhaltet kurzfristig verfügbare Vermögensgegenstände, so zum Beispiel Materialbestände, Kassenbestände, Forderungen oder Wertpapiere.
\\
Die Passiva bezeichnen die Herkunft der Mittel, also die Ansprüche der verschiedenen Kapitalgeber an das Unternehmensvermögen. Ansprüche von Gläubiger, die zum Beispiel aus Krediten oder Rechnungen entstanden sein können, bezeichnet man als Fremdkapital. Die Ansprüche der Unternehmer werden als Eigenkapital bezeichnet.\cite{bilanz}
\\
In einer Bilanz stellen Aktiva und Passiva stets dieselbe Wertgesamtheit dar, was in der Bilanzgleichheit Ausdruck findet. Auch fallen Rückstellungen, also Verbindlichkeiten, die hinsichtlich ihres Bestehens oder der Höhe ungewiss sind \cite{rueckstellung}, aber mit hoher Wahrscheinlichkeit eintreten und Verbindlichkeiten in die Kategorie der Passiva. Rückstellungen können zum Beispiel Steuerrückstellungen oder Pensionsrückstellungen sein. Zusätzlich zu beachten ist die sogenannte goldene Bilanzregel, welche fordert, dass langfristig an das Unternehmen gebundene Güter, also das Anlagevermögen, durch langfristiges Kapital, also vor allem Eigenkapital gedeckt wird. Das Umlaufvermögen kann hierbei auch durch Fremdkapital gedeckt sein. Eine Zuordnung der einzelnen Teile der Aktiva und Passiva ist hierbei nur rein rechnerisch und materiell möglich.\cite{bilanzregel}
\\
Dazu zeigt die Bilanz die Stabilität des Unternehmens und gibt wichtige Kennzahlen, wie Eigenkapitalquote (= Verhältnis von Eigenkapital zum Gesamtkapital) und Verschuldungsgrad, für das Unternehmen selbst als auch Investoren und Kreditgebern.\cite{businesson} Außerdem ist die Bilanz nötig um erkennen zu können, woher das Geld kommt und wohin es investiert wird. Die Bilanz ist satzungsmäßig und ermöglicht Transparenz der Finanzen eines Unternehmens.

\subsection{Profit \& Loss Statement, P\& L (Gewinn- und Verlustrechnung)}
Die Gewinn- und Verlustrechnung ist eine Gegenüberstellung von Erträgen und Aufwendungen innerhalb eines festgelegten Zeitraums, wie zum Beispiel eines Geschäftsjahres. Ziel der Gewinn- und Verlustrechnung ist es den Jahresüberschuss bzw. Fehlbetrag zu ermitteln und zu darzustellen aus welchen Quellen der unternehmerische Erfolg bzw. Misserfolg resultiert. Nach dem Handelsgesetzbuch kann die Gewinn- und Verlustrechnung mittels zwei verschiedener Verfahren aufgestellt werden. Das Erste ist das Umsatzkostenverfahren. Bei diesem Verfahren wird von den Erlösen die Kosten des Umsatzes und die Kosten der Funktionsbereiche abgezogen.
\\
Bei dem zweiten Verfahren, dem Gesamtkostenverfahren verrechnet man den Erlös mit Bestandsänderungen und zieht dann die gesamten Kosten ab. Das Gesamtkostenverfahren gliedert den Betriebsbezogenen Aufwand nach den verschiedenen Aufwandsarten, wie zum Beispiel Personalaufwand, Materialaufwand und Abschreibungsaufwand.
\\
Das Umsatzkostenverfahren bietet im Vergleich zum Gesamtkostenverfahren den Vorteil, dass die Aufwendungen stärker nach Verantwortungsbereichen aufgegliedert werden. Daher ist es möglich den Erfolg einzelner Bereiche wie Produkten, Produktgruppen oder Absatzwegen effizienter zu analysieren. Außerdem muss keine Inventur durchgeführt werden, wenn Bestandsveränderungen festgestellt werden sollen.
Jedoch ist das Umsatzkostenverfahren nicht ohne weiteres in die doppelte Buchführung integrierbar und muss durch zusätzliche Statistiken ergänzt werden.\cite{umsatzkosten}
\\
Die Gewinn- und Verlustrechnung ist, wie die Bilanz, satzungsmäßig und zusätzlich ein Instrument der Jahresabschlussanalyse. Dazu gibt es Investoren und Kreditgebern Informationen über vergangene finanzielle Leistung und hilft bei der Vorhersage zukünftiger Leistung, sowie Bewertung von Cash-Generierung.

\subsection{Cash Flow Statement (Kapitalflußtechnung)}
Die Kapitalflussrechnung ist ein eigenständiges Instrument zur Darstellung der Finanzlage eines Unternehmens. Darüber hinaus stellt es neben die Bilanz und die Gewinn- und Verlustrechnung, die Vermögens- und Ertragslage des Unternehmens dar. \cite{coenenberg2001kapitalflussrechnung} Kapitalfluss ist definiert als Überschuss der regelmäßigen betrieblichen Einnahmen über die regelmäßigen laufenden betrieblichen Ausgaben. Ziel der Kapitalflussrechnung ist es, Transparenz über die Zahlungsmittelflüsse des Unternehmens zu schaffen. Außerdem ist die Zahlungsfähigkeit besser zu analysieren als bei der Gewinn- und Verlustrechnung. Als Zahlenbasis dafür werden die Bilanz und die Gewinn- und Verlustrechnung genutzt. Es kann zwischen Kapitalflüssen aus der operativen Tätigkeit, der Investitionstätigkeit und der Finanzierungstätigkeit unterschieden werden. \cite{kapitalfluss}
\\ Zur Berechnung des Kapitalflusses sind zwei Methoden gebräuchlich. Die verbreitetste Methode ist die Indirekte, welche aus dem Jahresüberschuss die nicht zahlungswirksamen Erträge und Aufwendungen herausgerechnet. Nicht zahlungswirksame Erträge sind beispielsweise Auflösungen von Rückstellungen, Entnahmen aus Rücklagen oder Bestandserhöhungen anfertigen und unfertigen Erzeugnissen. Unter nicht zahlungswirksamen Aufwendungen versteht man unter anderem Einstellungen in die Rücklagen, Abschreibungen oder Bestandsminderung anfertigen und unfertigen Erzeugnissen. Weniger verbreitet ist die direkte Methode. Diese bildet erst die Summe aller zahlungswirksamen Erträge und subtrahiert davon die Summe aller zahlungswirksamen Aufwendungen. Hinter den zahlungswirksamen Erträgen verbergen sich Einzahlungen aus Umsätzen bzw. Forderungen, sowie Desinvestitionen, Kreditaufnahmen und Eigenkapitaleinlagen. Zu den zahlungswirksamen Aufwendungen zählt man Auszahlungen für Personal, Material und Verbindlichkeiten, Investitionen, Eigenkapitalentnahme und Kredittilgung.\cite{cashflow}
\\
Die Hauptzwecke der Kapitalflussrechnung sind die Bereitstellung von Informationen über Liquidität, Solvenz und Fähigkeiten des Unternehmens in zukünftigen Umständen, sowie Veränderungen der Vermögenswerte, Schulden und Eigenkapitals. Die Kapitalflussrechnung ermöglicht auch eine bessere Vergleichung der Leistungsfähigkeit zwischen verschiedenen Unternehmen, da Auswirkungen unterschiedlicher Rechnungslegungsmethoden eliminiert werden. \cite{coenenberg2001kapitalflussrechnung}
